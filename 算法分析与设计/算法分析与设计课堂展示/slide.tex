\documentclass{beamer}
\usepackage{ctex, hyperref}
\usepackage[T1]{fontenc}

% other packages
\usepackage{latexsym,amsmath,xcolor,multicol,booktabs,calligra}
\usepackage{graphicx,pstricks,listings,stackengine}

\usepackage{multirow}
\usepackage{algorithm}
\usepackage{algpseudocode}
\usepackage{amssymb}
\usepackage{amsfonts}
\usepackage{mathrsfs}

\author{麦骏}
\title{分支限界法总结}
\subtitle{课堂展示}
\institute{信息科学技术学院}
\date{2022年6月8日}
\usepackage{JNU}

% defs
\def\cmd#1{\texttt{\color{red}\footnotesize $\backslash$#1}}
\def\env#1{\texttt{\color{blue}\footnotesize #1}}
\definecolor{deepblue}{rgb}{0,0,0.5}
\definecolor{deepred}{rgb}{0.6,0,0}
\definecolor{deepgreen}{rgb}{0,0.5,0}
\definecolor{halfgray}{gray}{0.55}

\lstset{
    basicstyle=\ttfamily\small,
    keywordstyle=\bfseries\color{deepblue},
    emphstyle=\ttfamily\color{deepred},    % Custom highlighting style
    stringstyle=\color{deepgreen},
    numbers=left,
    numberstyle=\small\color{halfgray},
    rulesepcolor=\color{red!20!green!20!blue!20},
    frame=shadowbox,
}

\begin{document}
    
    \kaishu
    \begin{frame}
        \titlepage
        \begin{figure}[htpb]
            \begin{center}
                \includegraphics[width=0.4\linewidth]{pic/jnu.png}
            \end{center}
        \end{figure}
    \end{frame}

    \begin{frame}{分支限界法}
        \begin{itemize}
            \item 搜索算法主要优化在于剪枝力度。
            \item 分支限界法的精髓在于搜索树子树内上下界函数的确定。
            \item 当加入优先队列优化时,每次会选择一个子树可能最优的节点进行扩展。
            \item 但上下界相等时,说明已经获得子树内最优的答案。又因为每次选择的节点为全局可能最优的节点。那么当上下界相等时,该节点就是全局最优节点。
            \item 若需要求最小值,那么子树下界作为优先级,子树上界用来更新答案。若需要求最大值,则子树上界作为优先级,下界用来更新答案。
        \end{itemize}
    \end{frame}

    \begin{frame}{迭代加深搜索}
        \begin{itemize}
            \item 有一个长度为$n$全排列$p_i$.每次可以将某个前缀reverse,求最少几次操作可以将排列升序排序.
            \item $n \leq 25$.
        \end{itemize}
    \end{frame}
    
    \begin{frame}{迭代加深搜索}
        \begin{itemize}
            \item 若采用分支限界法,则子树下界可以定义为$p_i \neq p_{i+1}$的位置的个数,而子树上界可以贪心求得。
            \item 子树上界收敛速度慢,时空的浪费严重。
        \end{itemize}
    \end{frame}
    
    \begin{frame}{迭代加深搜索}
        \begin{itemize}
            \item 观察到答案与子树上界都是$O(n)$,可以从小到大枚举答案进行dfs搜索。
            \item 每次剪枝剪去子树下界大于枚举的答案的节点。
        \end{itemize}
    \end{frame}
    
    \begin{frame}{总结}
        \begin{itemize}
            \item 以上算法主要通过对于子树上下界估值进行剪枝,上下界确定的精确度决定了剪枝的力度。
            \item 另外,当上下界拥有较为特殊的性质时,可以调整搜索方式,进行优化。
        \end{itemize}
    \end{frame}
    
    \begin{frame}
        \begin{center}
            {\Huge\calligra Thanks!}
        \end{center}
    \end{frame}

\end{document}